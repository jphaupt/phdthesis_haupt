\chapter{Explicitly Correlated Methods}
\label{chap:explicit}
\todo{...}

Chapter will be largely based on two reviews on explicitly correlated methods \cite{gruneisPerspective2017,hattigExplicitly2012}

first discuss historical context

\section{The Cusp Conditions}
\label{sec:cusp}

cusp review \cite{kurokawaChapterTwoGeneral2016}

e-n and e-e cusp conditions due to kato\cite{katoEigenfunctionsManyparticleSystems1957a}

generalised \cite{packCuspConditionsMolecular1966}

behaviour of Phi in the vicinity of coalescence


\todo{mention STOs satisfy cusp}
\todo{mention cusp relationship with dynamical correlation}

\section{Hylleraas Methods}

earliest explicitly correlated method is when Hylleraas introduced a wave function form containing polynomial in r12
% \cite{hylleraasNeue1929}

\todo{Hylleraas method}

Hyl-CI methods, using short review \cite{largo-cabrerizoHylleraasCI1987}, though many different approaches have been proposed
\cite{jamesGround1933,kol/osAccurate1964,perkinsAtomic1968,perkinsAtomic1969,simsCombined1971,simsOneCenter1971,claryHylleraastype1977,claryCIHylleraas1976}
% ,kol/osAccurate1964
\todo{Hyl-CI methods...}

difficult integrals, why it's not used, etc.

\todo{...}

\section{An Overview of F12/R12 Methods}
\todo{... see review papers }

\todo{make sure to include some excited state information}

\section{The Transcorrelated Method}
\label{sec:tc}
\todo{brief recap of Boys-Handy method, some extensions by other people, and how it is used in our group}

Hirschfelder first introduced a similarity transform method \cite{hirschfelderRemoval1963}

\todo{also mention biorthogonal orbitals etc.}

\subsection{The Method of Boys and Handy}
\todo{Also cite Werner and quantum computing people}

\subsection{Modern Resurgence}

\subsection{Comparison to F12/R12}
