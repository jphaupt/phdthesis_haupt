\chapter{Introduction}
  \label{chap:intro}
\todo{...}

\todo{Basic introduction}
\todo{Discuss how impossible it is to solve basically everything, the CMP/chemistry "theory of everything" etc.}
\todo{Discuss models, ab initio methods (see Mona's 2Q notes)}

\section{Overview of the Thesis}

This dissertation fits into the field of electronic structure theory, the branch of quantum chemistry concerned with the description of electrons and their correlation inside molecules and materials. More specifically, this dissertation focuses on high-accuracy (but often high-cost) \emph{ab initio} methodologies, especially \gls{FCIQMC} and \gls{TC}. As such, I will only be discussing small systems of consisting of only a few atoms, as they are tractible with a full, all-electron treatment with these methods. In principle, these methods should be able to be embedded \todo{embedding citation(s)} into more large-scale calculations using multiscale techniques, but this is outside the scope of my work. Nevertheless, the work herein is focused on methodologies, and not on particular physical systems.

The outline of the dissertation is as thus:
\begin{itemize}
    \item Chapter \ref{chap:intro} (this chapter) provides a basic overview of electronic structure theory methods and some of its principle concepts.
    \item Chapter \ref{chap:qmc} provides a basic introduction to \gls{QMC} and how it relates to \gls{FCIQMC}.
    \item Chapter \ref{chap:explicit} reviews the current works in so-called ``explicitly correlated'' methods, notably the well-established R12/F12 and the recently-reinvigorated \gls{TC}.
    \item Chapter \ref{chap:opt} discusses optimization of Jastrow factors in the context of \gls{TC}.
    \item Chapter \ref{chap:binding} discusses an extension of the methods in the previous chapters to ensure size consistency and success when targeting strongly multi-reference problems.
    \item Chapter \ref{chap:pytchint} provides an overview of the software \pytchint developed in the group for evaluation of \gls{TC} integrals.
    \item Finally, chapter \ref{chap:sumandout} provides a review and an outlook for the field.
\end{itemize}

% $\bra{}$
% $\one$
% $\hat H$

% introduction and context
% The quantum many-body problem, the TISE, in principle have everything but it's exceedingly difficult in practice, strong vs weak correlation, need for interesting materials and even small molecules to understand e.g. Haber-Bosch process and nitrogen fixation, mention multiscale methods (see Mona's notes?)
% scope of the thesis
% electronic structure theory
% "all models are wrong, but some of them are useful" (G. E. P. Box, N. R. Draper)
% Model Hamiltonians (Ashcrof & Mermin)
% Hartree-Fock Theory
% correlation energies static vs dynamic, explicit correlation
% post-HF methods

% small preface about the work, publication(s)

% \one
