\cleardoublepage
\begin{otherlanguage}{ngerman}
\addchap{Zusammenfassung}

Die Transkorrelierten (TC) Methode ist eine Methode in der Quantenchemie, die in den letzten Jahren an Bedeutung gewonnen hat. In diese Methode wird eine Ähnlichkeitstransformation auf den elektronischen Hamiltonoperator angewendet, um Korrelationseffekte, insbesondere der dznamischen Korrelation, durch explizite Behandlung analytisch bekannter Eigenschafen des Hamiltonoperator zu erfassen. Diese Methode wurde bereits mit der ,,Full Configuration Interaction Quantum Monte Carlo'' (FCIQMC) Methode kombiniert, einem effiyienten stochastischen Methode zur Lösung der Schrödingergleichung und zur Berechnung physikalischer Observablen. Diese Dissertation untersucht diese Methoden, insbesondere die TC Methode, weiter.

Nach einer Wiederholung der Grundlagen der Quantenchemie, wird die Verwendung flexibler Jastrow-Faktoren, die in der Literatur Variational Quantum Monte Carlos (VMC) bekannt sind, vorgestellt. Damit minimieren wir die Varianz der TC-Referenzenergie.Es zeigt sich, dass dies zu einer schnellen Basissatzkonvergenz führt und eine Genauigkeit erreicht, für die herkömmliches FCIQMC viel größere Basissätze erfordern würde. Außerdem kompaktiert dieses Mimimierungsverfahren auch die Wellenfunktion, was eine effizientere FCIQMC-Berechnung ermöglicht. Als nächstes erweitern wir die Methode für Probleme mit starken Multireferenzproblemen, insbesondere indem wir die Dissoziation des N$_2$ als Stresstest verwenden. Wir zeigen die Notwendigkeit eines Multireferenz-Jastrow-Faktor-Ansatzes und minimieren daher die Varianz eines Multireferenzzustands. Dies wird nachweislich günstige, größenkonsistente Energien wiederherstellen und gleichzeitig die schnelle Basissatzkonvergenz von TC erreichen. Da nach der Jastrow-Faktor-Optimierung eine Multireferenztechnik (FCIQMC) verwendet wird, erhöht sich der rechnerische Aufwand für die Methode darüber hinaus nicht, wenn man die konventionelle (nicht-TC-)Form deselben Multireferenztechnik wie beim TC-Ansatz verwendet.

Abschließend untersuchen wir die Möglichkeit, vereinfachte Jastrow-Faktoren zu kontruieren, um das bisherige Optimierungsverfahren für die TC-Methode, das rechenintensiv sein kann, zu verbessern oder sogar ganz zu umgehen. Einerseits zeigen wir, dass parameterfreie Jastrow-Faktoren aufgrund der Fehlerkompensation zu schlechten absoluten Energien, aber relativen Energien führen können. Andererseits ergeben Jastrow-Faktoren, die für Atome optimierte waren und wiederverwendete für Moleküle, sowohl genaue absolute als auch relative Energien. Dies eröffnet die Möglichkeit, Jastrow-Faktoren für Atome im gesamten Periodensystem zu optimieren und sie in einer Datenbank zu speichern, die für größere Moleküle abgeragt werden kann, was die Skalierbarkeit der Methode unterstützt.

\end{otherlanguage}

\addchap{Abstract}
The transcorrelated (TC) method is a technique in electronic structure theory that has recently been gaining momentum. In it, a similarity transformation is applied to the electronic Hamiltonian to capture effects of electron correlation, particularly dynamical correlation, by explicit treatment of analytically-known properties of the Hamiltonian near coalescence points. This has already been combined with the full configuration interaction quantum Monte Carlo (FCIQMC) method, an efficient stochastic approach to solving the electronic Schrödinger equation and calculating physical observables. This dissertation further explores these methods, particularly TC.

After a recapitulation of the core concepts of wave function and quantum Monte Carlo methods, we introduce the use of flexible Jastrow factors familiar in the variational Monte Carlo (VMC) literature to minimise the variance of the TC-reference energy. This is shown to result in a rapid basis-set convergence, reaching accuracy for which conventional FCIQMC would require much larger basis sets. Moreover, this minimisation procedure is shown to also compactify the wave function, allowing for more efficient sampling in FCIQMC. We next extend the methodology for problems of strongly multireference problems, notably using the dissociation of the nitrogen dimer as a stress test. We illustrate the need for a multireference Jastrow-factor ansatz, and hence minimise the variance of a multireference state. This is shown to recover favourable, size-consistent energies while maintaining the rapid basis-set convergence that comes with TC. As a multireference technique (FCIQMC) is used after optimisation, it moreover does not increase the computational scaling of the method to use the conventional (non-TC) form of the same multireference technique as the TC ansatz.

Finally, we explore the possibility of constructing simplified Jastrow factors in order to improve or possibly wholly bypass the optimisation procedure so far for the TC method, which can be computationally expensive. We show that parameter-free Jastrow factors can result in poor absolute energies, but favourable relative energies thanks to error cancellation, whereas Jastrow factors optimised for atoms and reused for molecules result in both accurate absolute and relative energies. This opens the possibility of optimising Jastrow factors for atoms across the periodic table and storing them in a database, which can be queried for larger molecules, thereby aiding the scalability of the method.
