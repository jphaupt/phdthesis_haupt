\chapter{Monte Carlo Methods}
\label{chap:qmc}
\todo{...}
\todo{look especially at thijssen and the foulkes review for these sections}

\gls{MC} methods are a class of numerical methods that use random sampling to numerically solve problems. It has found applications in an impressive range of fields, from physics to finance.\todo{citation} It is particularly useful for problems with high dimensionality, where deterministic methods are often impractical. In quantum chemistry and physics, since a `dimension' can refer to any degree of freedom, high-dimensional problems are commonplace, and so \gls{MC} methods are a natural choice.

While the name \gls{MC} was coined by Stanislav Ulam, after the famous casino in Monaco,\todo{citation} the foundational concept was already developed in the 18th century by the French mathematician Georges-Louis Leclerc, Comte de Buffon. As one of the earliest example applications, in the Buffon needle problem, one can randomly toss needles onto a lined sheet of paper and determine $\pi$.\todo{citation}

\section{Classical Monte Carlo Methods}

We start out discussion with classical \gls{MC} methods. We consider the classical textbook problem of calculating the value of $\pi$, then we give it a more rigorous framework.

\subsection{A Very Bad Game of Darts}

\subsection{A More Mathematical Description}

\subsection{The Metropolis-Hastings Algorithm}

\section{Going Quantum: Variational Monte Carlo}
\todo{make sure to mention the Jastrow factor}

\section{Diffusion Monte Carlo}
\todo{main point here is just its similarity to FCIQMC}

\section{The Fermion Sign Problem}
\todo{make sure to mention the fixed node approximation here}

\section{QMC Meets Electronic Structure: the FCIQMC Algorithm}

\subsection{The Sign Problem in FCIQMC}

\subsection{The Initiator Approximation}

\subsection{Reduced Density Matrix Sampling}
