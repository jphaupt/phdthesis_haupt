\chapter{Summary and Outlook}
  \label{chap:sumandout}

Transcorrelation has recently enjoyed somewhat of a revival since its early development in 1960s, particularly when combined with other methods such as \gls{CC}, \gls{DMRG}, or \gls{FCIQMC}. The topic of this dissertation has been to further explore modern developments of the \gls{TC} method, especially applied to FCIQMC. Both of these methods combine ideas from \gls{QMC} and applies them to wave function methods. FCIQMC stochastically samples the (ground state) wave function of a system in analogy to \gls{DMC} to give \gls{FCI}-accuracy. The TC method adopts the Jastrow factor typically encountered in \gls{VMC} to correlate the electrons and explicitly take into account analytically-known behaviour at coalescence, a task difficult for conventional methods. Combining the two results in an essentially-exact stochastic method with rapid convergence towards the \gls{CBS} limit.

The primarily contributions to this field represented in this dissertation are threefold. First, in \autoref{chap:opt}, using flexible Jastrow factors with VMC to minimise the variance of the reference energy is proposed and shown to result in not only highly-accurate energies with chemically-accurate atomisation energies with only the \vtz basis set (compared to \vxz{5} for non-TC-FCI), but it also compactifies the wave function, making it better suited to sampling methods such as FCIQMC.

Second, in \autoref{chap:binding} we have shown that the TC method was not able to accurately describe problems of strongly multireference character, such as the dissociation of N$_2$, despite using FCIQMC after Jastrow-factor optimisation. Modifying the TC ansatz to accommodate multireference wave functions, we show that we can still benefit from using the TC method even in such challenging problems. Moreover, this introduces additional flexibility in the optimisation procedure, allowing us to tailor our Jastrow factor for specific states. We showed that this can be used to determine excited state energies with high accuracy.

Finally, in \autoref{chap:universal} we consider alternative forms for the Jastrow factor, either allowing much faster optimisation or bypassing it altogether, implying better suitability for larger systems. We considered two different categories of such Jastrow factors. The first, the ``universal'' Jastrow factors were in a sense minimal, insofar as primarily handling known analytical behaviour of the wave function near coalescence points. These Jastrow factors were shown to sometimes result in nonvariational energies, but they did nevertheless exhibit favourable error cancellation. This is suggestive for future studies to prevent the nonvariational energies from occurring, possibly by optimising the cutoff range of the Jastrow factor terms. The second category of Jastrow factors were the atomic Jastrow factors, wherein the Jastrow factors are optimised for each atom in the system, and then these Jastrow factors are used in the molecule, only reoptimising a small portion of the parameters. This still resulted in clearly-superior basis-set convergence compared to conventional methods, while not requiring as much computational effort as the full optimisation. Moreover, the energies we found were all above the CBS limit. In principle, this opens the possibility of optimising sophisticated Jastrow factors for atoms across the periodic table, storing them in a database, and querying it when needed in a calculation, in analogy to what is already done with basis sets.

The methods proposed in this thesis have already been applied in other studies, such as applying TC to larger atoms\supercite{filip_2ndrow} or combining TC with \glspl{ECP} (where the FCIQMC-Jastrow approach of \autoref{chap:binding} can also be used).\supercite{simulaEcp} Other potential avenues of research include applying the methodologies to larger systems, such as solids, or embedding with the TC method. Combining the optimisation procedure outlined in \autoref{chap:opt} with the active space multireference approach in \autoref{chap:binding}, we could conceivably optimise biorthogonal orbitals in a transcorrelated CASSCF-like procedure for high accuracy within an active space. This might also be combined with simpified Jastrow factors such as those presented in \autoref{chap:universal} and deterministic optimisation as in \autoref{chap:pytchint} to allow for more robust but scalable calculations.

Despite having its roots in the 1960s, the methodology is still in its infancy, and as a result the most productive step forward would likely be to simply apply the methodology to new problems of chemical interest. This would necessarily lead to new methods, computational techniques, and code optimisations.
