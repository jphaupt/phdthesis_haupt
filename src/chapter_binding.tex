\chapter{The Transcorrelated Method for Multi-Reference Problems}
  \label{chap:binding}
\todo{...}

This chapter is based in large part on the following paper:\\
\fullcite{haupt_sizeconsistent}

Images have been reused from this paper (with permission).

\section{Introduction}

In this chapter, we apply the new framework for the transcorrated method described in chapter \ref{chap:opt} to problems of multireference character and find these methods may yield non-physical results. We propose an updated workflow wherein we use conventional FCIQMC as input to Jastrow optimisation for TC-FCIQMC. Results suggest size-consistent results and rapid basis set convergence compared to conventional methods, with the binding curve of N$_2$ at \avtz being within chemical accuracy of experiment.

\section{Motivation}

Consider the methodology outlined in chapter \ref{chap:opt}. If we again use the same Jastrow factors,
\todo{write Jastrow factor down again}
with the same objective function,
\todo{write variance of reference with a single(!) reference}
\begin{equation}
\label{eq:varref_hf}
TODO
\end{equation}
and we use the workflow to calculate points along the binding curve of N$_2$, we find a non-physical ``bump'', as shown in figure \ref{fig:binding-dip}.

\begin{figure}[htbp]
    \centering

    \caption{\todo{figure showing a plot of the binding curve(s) with a single reference!}}
    \label{fig:binding-dip}
\end{figure}

\todo{problem description, including the ``dip''}

\section{Size Consistency}

\subsection{Jastrow Factor Forms}

\subsection{Multireference Ansatzes}

\todo{Present: "circular" approach, CASCI, CASSCF}

% \todo{present the bad data :( -- mention might hint at needing the core (I think Andreea's paper corroborated this)}

\section{Transcorrelated Trial Wavefunction}

\section{Results}

\subsection{Binding Curves}
\todo{mention that we fix the binding curves}
\todo{also include a table about just ``how size-consistent'' the results actually are.}

\subsection{Excitation Energies}
\todo{mention can target the exact state(s) in question}

\section{Conclusion and Outlook}
\todo{make sure to mention the cutoff analysis and constructing Jastrow factors from atomic Jastrow factors. Mention ECPs, since it's already been submitted...}
\todo{TC-CAS...}
\todo{self-consistent TC (i.e. feed back into itself)}
