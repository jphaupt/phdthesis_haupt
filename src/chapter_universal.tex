\chapter{(Quasi-)Universal Jastrow Factors}
\label{chap:universal}

The contents of this chapter are planned to be expanded for a future publication. Some images may be repeated there.

\section{Introduction}

\todo{describe motivation (ease of use, scalability, generalisability)}

\section{Theory}

\todo{describe how to construct them, the different forms (Fournais, r12 only, atomic, quasi-atomic)}

\section{Results}
\todo{N2 binding curve, some atomisation energies (I guess C2, N2 and CN are enough)}

\section{Conclusion and Outlook}
\todo{mention the shortcoming of not being able to target specific states like the previous methodology (except for quasi-atomic)}
\todo{mention possibility of combining methods, also using the orbital cusp correction for the Fournais factor}
\todo{...}

\todo{Also mention (maybe by then you even have data for) the Fournais Jastrow factor}
