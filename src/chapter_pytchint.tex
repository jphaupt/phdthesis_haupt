\chapter{The PyTCHInt Library}
\label{chap:pytchint}
\todo{...}

\todo{this might become an appendix, and instead we have a chapter about universal or quasi-universal Jastrow factors}

This chapter is based on the following software:\\
\fullcite{tchint}

The basis for how the library works is partially discussed in the following paper and its supplementary material: \\
\todo{Aron's tchint paper}

\todo{May want to move this chapter closer to the start, i.e. just before the two other ``original'' chapters, to illustrate the tools used for those. Not sure yet.}

\section{Introduction}

\section{Matrix Element Evaluation}
\todo{mention the different ``intermediate'' quantities, RI (no longer really used), xTC}

\section{Deterministic Optimisation}

\section{Performance}
\todo{not really sure how I'll write this section, since I don't really have a performance metric to go up against}
\todo{memory requirements, e.g. full TC, no-3-body, xTC}

\section{Interface}
\todo{NECI interface, M7 interface, pytchint and interactive HPC}

\section{Conclusion and Outlook}
