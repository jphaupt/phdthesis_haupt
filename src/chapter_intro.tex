\chapter{Introduction}
\label{chap:intro}

According to modern quantum theory, to fully describe\footnote{This neglects relavistic effects.} a system of $N_A$ atoms and $N_e$ electrons, we must solve the time-dependent Schr\"odinger equation,
\begin{equation}
    \label{eq:schrodinger}
    i \hbar \frac{\partial}{\partial t} \Psi(\bm{x}, t)
    = \hat{H} \Psi(\bm{x}, t)
\end{equation}
where $x=(\bm{x}, \sigma)$, $\bm r\in\mathcal{R}^{3(N_e+N_A)}$ is the spatial coordinates, and $\sigma$ are the corresponding spins. For atoms with atomic numbers $Z_I$ (positive charge $Ze$), and with $N_e=ZN_A$, the Hamiltonian operator $\hat H$ may be written

\begin{align}
\begin{split}
\label{eq:theory_of_everything}
\hat H =& -\sum_{I=1}^{N_A}\frac{\hbar^2}{2m_I} \nabla^2_I
+ \frac 1{4\pi\epsilon_0} \frac 12\sum_{I,J=1;I\neq J}^{N_A} \frac{Z_IZ_Je^2}{|\bm R_I-\bm R_J|} \\
&- \sum_{i=1}^{N_e} \frac{\hbar^2}{2m_e} \nabla^2_i
+ \frac 1{4\pi\epsilon_0} \frac 12\sum_{i,j=1;i\neq j}^{N_e} \frac{e^2}{|\bm r_i-\bm r_j|}
- \sum_{i=1}^{N_e} \sum_{I=1}^{N_A} \frac{Z_I e^2}{|\bm r_i-\bm R_I|}
\end{split}
\end{align}
where we have represented the nuclear coordinates by $\bm R_I$ and the electron coordinates by $\bm r_i$.

Since quantum chemistry and condensed matter sciences are in general concerned with nonrelativistic processes involving electrons and nuclei, this might be called the \emph{theory of everything}. Hence, we may be tempted to conclude this dissertation early, but in practice the evaluation of the Hamiltonian \eqref{eq:theory_of_everything} is impossible. First, there is no closed form solution of \eqref{eq:schrodinger} with this Hamiltonian. Second, numerical evaluation of $\hat H$ is far from trivial.

Let's consider a simple example, the argon atom. Say we want to solve this partial differential equation on a grid. Let's choose a simple, coarse $100\times 100\times 100$ grid. Then \emph{at each time step} we need to store $(100^3)^{23}=10^{138}$ values, corresponding to all the particle positions and the grid. Considering there are ``only'' $\sim 10^{80}$ atoms in the known universe, this is quite a tall order!\todo{get a citation for this number (look at ASTR403 notes)}

This system has only 18 electrons and 18 nuclei, a farcry from the $10^{23}$ or higher number of electrons in a typical condensed matter system, for example. Moreover, we have not taken into account floating point precision, or that we would need to calculate this for possibly many time steps. Clearly, drastic approximations and more sophisticated methods are required.

\section{Overview of the Thesis}

This dissertation fits into the field of nonrelativistic electronic structure theory, the branch of quantum chemistry concerned with the description of electrons and their correlation inside molecules and materials. More specifically, this dissertation focuses on high-accuracy (but often high-cost) \emph{ab initio} methodologies, especially \gls{FCIQMC} and \gls{TC}. As such, we will only be discussing small systems of consisting of only a few atoms, as they are tractible with a full, all-electron treatment with these methods. In principle, these methods should be able to be embedded \todo{embedding citation(s)} into more large-scale calculations using multiscale techniques, but this is outside the scope of my work. Nevertheless, the work herein is focused on methodologies, and not on particular physical systems.

The outline of the dissertation is as thus:
\begin{itemize}
    \item Chapter \ref{chap:intro} (this chapter) provides a basic overview of electronic structure theory methods and some of its principal concepts.
    \item Chapter \ref{chap:qmc} provides a basic introduction to \gls{QMC} and how it relates to \gls{FCIQMC}.
    \item Chapter \ref{chap:explicit} reviews the current works in so-called ``explicitly correlated'' methods, notably the well-established R12/F12 and the recently-reinvigorated \gls{TC}.
    \item Chapter \ref{chap:opt} discusses optimization of Jastrow factors in the context of \gls{TC}.
    \item Chapter \ref{chap:binding} discusses an extension of the methods in the previous chapters to ensure size consistency and success when targeting strongly multi-reference problems.
    \item Chapter \ref{chap:pytchint} provides an overview of the software \pytchint developed in the group for evaluation of \gls{TC} integrals.
    \item Finally, chapter \ref{chap:sumandout} provides a review and an outlook for the field.
\end{itemize}


\section{Principal Approximations}

As discussed, to make any progress in electronic structure theory, we must make use of approximations. Of course, we must always be cautious and suspicious when using these, and make sure they are valid for the systems in question. Thankfully, over the course of the last century, a cornucopia of different approximations have been developed to tackle the Schr\"odinger question, some of which will be discussed in this section.

\subsection{The Born-Oppenheimer Approximation}

The most important approximation used in electronic structure theory is the \gls{BOA}.\todo{BOA citation} The \gls{BOA} relies on the fact that the nuclei are much heavier than the electrons, with the mass of a single proton being almost 2000 times the mass of an electron. As an intuitive picture, we may think of the nuclei as moving much slower than the electrons, which can adapt themselves to the instantaneous positions of the nuclei. In mathematical terms, this means we can take the total wave function to be a product of its nuclear and electronic components,
\begin{equation}
    \Psi_\mathrm{total} = \Psi_\mathrm{nuc} \Psi_\mathrm{elec}.
\end{equation}

Notice that the first two terms of \ref{eq:theory_of_everything} are independent of the electronic coordinates and, ipso facto, have no effect on $\Psi_\mathrm{elec}$. This leads to the \emph{electronic Hamiltonian} under the \gls{BOA}, which can be written in atomic units as
\begin{equation}
\label{eq:elec_hamiltonian}
\hat H_\mathrm{elec} = -\sum_{i} \frac{1}{2} \nabla_i^2 - \sum_{i,I} \frac{Z_I}{r_{iI}} + \sum_{i\gt j} \frac{1}{r_{ij}},
\end{equation}
where we have simplified notation by using miniscule roman letters for the electrons and capital roman letters for the nuclei, and $r_{ij}=|\mathbf{r}_i-\mathbf{r}_j|$ and $r_{iI}=|\mathbf{r}_i-\mathbf{R}_I|$.

In the language of second quantisation, equation \ref{eq:elec_hamiltonian} can be written as
\todo{still haven't finished this eqn}
\begin{equation}
\label{eq:elec_hamiltonian_2q}
\hat H_\mathrm{elec} = \sum_{ij\sigma} h_{ij}a_{i\sigma}^\dag a_{j\sigma}+\frac 12 \sum_{ijkl\sigma\tau} g_{ijkl}a_{i\sigma}^\dag a_{j\tau}^\dag a_{k\tau}a_{l\sigma},
\end{equation}
where $\sigma,\tau\in\{\uparrow,\downarrow\}$ are spin indices, and $a_{i\sigma}$ ($a_{i\sigma}^\dag$) is the annihilation (creation) operator for an electron on spin-orbital $i$ with spin $\sigma$. These must obey the anti-commutation relation
\begin{equation}
    \{a_{i\sigma},a_{j\tau}^\dag\}
    \mathdef
    a_{i\sigma}a_{j\tau}^\dag + a_{j\tau}^\dag a_{i\sigma}
    =\delta_{ij}\delta_{\sigma\tau}
\end{equation}
so that the electrons satisfy the Pauli exclusion principle.

In equation \ref{eq:elec_hamiltonian_2q}, $h_{ij}$ is a one-body integral,
\begin{equation}
\label{eq:hij}
h_{ij} = \int \phi_i^*(\mathbf{r})\left(-\frac 12 \nabla^2 - \sum_I \frac{Z_I}{|\mathbf{R}_I-\mathbf r|}\right)\phi_j(\mathbf{r})\mathrm{d}\mathbf{r}
\end{equation}
and $g_{ijkl}$ is a two-body integral,
\begin{equation}
g_{ijkl} = \int \phi_i^*(\mathbf{r})\phi_k^*(\mathbf{r}')\frac{1}{|\mathbf{r}-\mathbf{r}'|}\phi_j(\mathbf{r}')\phi_l(\mathbf{r})\mathrm{d}\mathbf{r}\mathrm{d}\mathbf{r}',
\end{equation}
with a spin-orbital basis $\{\phi_i(\mathbf{r})\}$.

Finally, to treat the nuclear Hamiltonian, we neglect the first term of equation \eqref{eq:theory_of_everything}, and treat the second term as approximately constant, so $\hat H_\mathrm{nuc} = \sum_{IJ} Z_IZ_Jr_{IJ}$. Thus, we are left with the problem of solving the electronic structure problem, which is the subject of this dissertation.

Note that while chemists and physicists typically talk about the positions of nuclei in a molecule or solid as if they are fixed in place, as will be done in this dissertation, this is really a colloquialism. If the nuclei had an exact position and zero kinetic energy, the \gls{BOA} would, ipso facto, be in direct contradiction of the Heisenberg uncertainty principle. Instead, on the timescale of the electrons, due to the much higher mass of nuclei, in the \gls{BOA} we treat the nuclei as approximately localised in a state in which their motion is much slower than that of the electrons (but, importantly, not zero). This keeps the approximation from being in conflict with the fundamental postulates of quantum theory.

The \gls{BOA} is an immensely practical tool as it substantially simplifies our equations, and in many applications it is an excellent approximation. It will be a fundamental assumption throughout the rest of this dissertation, though it need not always be valid in all of quantum chemistry.

While we have done a lot to drastically reduce the complexity of equation \eqref{eq:theory_of_everything}, equation \eqref{eq:elec_hamiltonian_2q} is still intractible for large system sizes, scaling combinatorially with the size of the Hilbert space, as function of the system size $N_e$ (henceforth $N$), and the basis set size $M$. Hence, in addition to using a smaller basis, we still need additional approximations sophisticated methodologies.


\subsection{Core Electrons}



\todo{...}

\subsection{Model Hamiltonians}

In 1929, the surrealist artist Ren\'e Magritte displayed a now-famous painting of a pipe with the caption \emph{Ceci n'est pas une pipe} (French for ``This is note a pipe''). It was meant to depict the idea that the painting itself is in a way treacherous: it may appear to be a pipe, but you cannot stuff it or smoke from it, as it is a representation of a pipe.

In a similar way, physicists and chemists often use \emph{model Hamiltonians}, which do away with aspects of equation \eqref{eq:theory_of_everything} that are not expected to be relevant to the problem at hand, resulting in new Hamiltonians that may be considered a representation of equation \eqref{eq:theory_of_everything}, much like Magritte's painting.\footnote{We might further argue that all of science can be described this way, as scientific models are always ``mere'' representations of Nature, and is not Nature itself.} Compared to \emph{ab initio} Hamiltonians like equation \eqref{eq:elec_hamiltonian_2q}, model Hamiltonians are generally much simpler, but depend on parameters whose values we may not necessarily know a priori.

\todo{...}

\todo{all models are wrong but some are useful quotation, a map is not the territory, ceci n'est pas une pipe}

\section{The Hartree-Fock Method}

\todo{...}

\section{Post-Hartree-Fock Methods}

\todo{...}
\todo{also discuss static vs dynamic correlation}


% introduction and context
% The quantum many-body problem, the TISE, in principle have everything but it's exceedingly difficult in practice, strong vs weak correlation, need for interesting materials and even small molecules to understand e.g. Haber-Bosch process and nitrogen fixation, mention multiscale methods (see Mona's notes?)
% scope of the thesis
% electronic structure theory
% "all models are wrong, but some of them are useful" (G. E. P. Box, N. R. Draper)
% Model Hamiltonians (Ashcrof & Mermin)
% Hartree-Fock Theory
% correlation energies static vs dynamic, explicit correlation
% post-HF methods

% small preface about the work, publication(s)

% \one
