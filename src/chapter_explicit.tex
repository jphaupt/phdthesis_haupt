\chapter{Explicitly Correlated Methods}
\label{chap:explicit}

\todo{If I need to make this chapter longer, maybe also include a discussion about what electron correlation actually is, like in the Kong review.}

With the advent of modern computers combined with a vast array of sophisticated algorithms from which to choose, \emph{ab initio} quantum chemistry has become a tremendously powerful tool, going beyond the study of small atoms, to molecules and solids, and are among the most effective and systematically improvable techniques to date. Nevertheless, convergence to the \gls{CBS} limit is notoriously slow.

In particular, consider the popular basis set family developed by Dunning and coworkers, \gls{cc-pV$X$Z}.\cite{dunningGaussian1989a,woonGaussian1993,woonGaussian1994,petersonBenchmark1994,wilsonGaussian1996}
The size of these basis sets scale as $M\in\mathcal{O}(X^3)$, and since for standard post-Hartree-Fock discussed in chapter \ref{sec:post-hf} we require four-index integrals, our computation time will scale at least as $t\in\mathcal{O}(X^{12})$.\cite{klopperR122007}

Meanwhile, the \gls{CBS} correlation error scales as $\epsilon\in\mathcal{O}(X^{-3})$ \cite{helgakerBasisset1997,halkierBasisset1998} or $\epsilon\in\mathcal{O}(M^{-1})$,\cite{klopperInitio1995} resulting in $t\in\mathcal{O}(M^{-\frac 14})$. Thus, the methods discussed so far come with the painful cost of requiring very large basis sets to approach high-accuracy results.

Explicitly correlated methods are a class of electronic structure methods specifically designed to address this unfavourable scaling, and the subject of this chapter, which is in large part based on two reviews, references \citenum{gruneisPerspective2017} and \citenum{hattigExplicitly2012}.

\todo{Maybe have a picture of basis set convergence? Maybe even just HF, compare F12 vs TC vs Hyl(?) vs standard}

\section{The Cusp Conditions}
\label{sec:cusp}

The slow convergence to the \gls{CBS} limit is due to behaviour of the wave function near coalescence points.

\todo{also mention it's not actually the coalescence itself, but a region around it; see one of the reviews (I forget which one)}

cusp review \cite{kurokawaChapterTwoGeneral2016}

e-n and e-e cusp conditions due to kato\cite{katoEigenfunctionsManyparticleSystems1957a}

generalised \cite{packCuspConditionsMolecular1966}

behaviour of Phi in the vicinity of coalescence


\todo{mention STOs satisfy cusp (nice discussion in the Kong review)}
\todo{mention cusp relationship with dynamical correlation}

\todo{figure of the cusp condition -- would be nice to think of other ways to add images}

\section{Hylleraas Methods}

earliest explicitly correlated method is when Hylleraas introduced a wave function form containing polynomial in r12\cite{hylleraasNeueBerechnungEnergie1929}
\todo{also mention Slater's early work on this? (I think this is explicitly correlated anyway) No one else seems to mention it and I can't seem to remember where I first found this reference...
nvm it is definitely cited elsewhere. see haettig review!}
\cite{slaterCentralFieldsRydberg1928}

\todo{Hylleraas method}

Hyl-CI methods, using short review \cite{largo-cabrerizoHylleraasCI1987}, though many different approaches have been proposed
\cite{jamesGround1933,kolosAccurate1964,perkinsAtomic1968,perkinsAtomic1969,simsCombined1971,simsOneCenter1971,claryHylleraastype1977,claryCIHylleraas1976}
\todo{Hyl-CI methods...}

difficult integrals, why it's not used, etc.

\todo{...}

\section{An Overview of F12/R12 Methods}
\todo{... see review papers }

brief mention of gaussian-type geminals\cite{boysIntegral1960,singerUse1960}, lots of work on refining and extending it.\cite{lesterGaussian1964,mitroyTheory2013,bukowskiNew1994,perssonAccurate1996,panGaussian1970,panElectron1972,szalewiczNew1982,szalewiczAtomic1983,wenzelAtomic1986}

\todo{make sure to include some excited state information}

\section{The Transcorrelated Method}
\label{sec:tc}
\todo{brief recap of Boys-Handy method, some extensions by other people, and how it is used in our group}

Hirschfelder first introduced a similarity transform method \cite{hirschfelderRemoval1963}

then was worked on by Boys and Handy \todo{cite}, to be discussed here

recent renewal of interest (lots and lots of citation) with various applications, including DMRG, quantum computing, etc.

\todo{also mention biorthogonal orbitals etc.}

\subsection{The Method of Boys and Handy}
\todo{Also cite Werner and quantum computing people}

\subsection{Modern Resurgence}

\subsection{Comparison to F12/R12}

discuss additive vs multiplicative ansatz

have some performance comparison plots, maybe can ask permission to reproduce from Evelin or someone else

also discuss many-body integrals for f12 vs at most 3-body for tc
