\chapter{Explicitly Correlated Methods}
\label{chap:explicit}

With the advent of modern computers combined with a vast array of sophisticated algorithms from which to choose, \emph{ab initio} quantum chemistry has become a tremendously powerful tool, going beyond the study of small atoms, to molecules and solids, and are among the most effective and systematically improvable techniques to date. Nevertheless, convergence to the \gls{CBS} limit is notoriously slow.

In particular, consider the popular basis set family developed by Dunning and coworkers, \gls{cc-pV$X$Z}.\cite{dunningGaussian1989a,woonGaussian1993,woonGaussian1994,petersonBenchmark1994,wilsonGaussian1996}
The size of these basis sets scale as $M\in\mathcal{O}(X^3)$, and since for standard post-Hartree-Fock discussed in chapter \ref{sec:post-hf} we require four-index integrals, our computation time will scale at least as $t\in\mathcal{O}(X^{12})$.\cite{klopperR122007}

Meanwhile, the \gls{CBS} correlation error scales as $\epsilon\in\mathcal{O}(X^{-3})$ \cite{helgakerBasisset1997,halkierBasisset1998} or $\epsilon\in\mathcal{O}(M^{-1})$,\cite{klopperInitio1995} resulting in $t\in\mathcal{O}(M^{-\frac 14})$. Thus, the methods discussed so far come with the painful cost of requiring very large basis sets to approach high-accuracy results.

Explicitly correlated methods are a class of electronic structure methods specifically designed to address this unfavourable scaling by explicitly including the interelectronic distance $r_{12}$, and is the subject of this chapter. As the R12/F12 family of methods is the most mature of the explicitly correlated electronic structure methods, many reviews focusing on this topic already exist. This chapter in particular is in large part based on references \parencite{klopperR122007,gruneisPerspective2017,hattigExplicitly2012,shiozakiMultireference2013}.

% \textcite{klopperR122007,gruneisPerspective2017,hattigExplicitly2012,shiozakiMultireference2013}.

% three reviews: references \citenum{klopperR122007}, \citenum{gruneisPerspective2017} and \citenum{hattigExplicitly2012}.


\section{The Cusp Conditions}
\label{sec:cusp}

Consider two charged point particles in a system described by the Hamiltonian of equation \eqref{eq:elec_hamiltonian_2q}. By the Schr\"odinger equation, the local energy
\begin{equation}
    E_L\mathdef \frac{H\Psi}{\Psi}
\end{equation}
must be constant in the exact solution. However, when these two particles coalesce, i.e. $r_{12}\to 0$, the Coulomb potential, $r^{-1}$, diverges. Thus, for the local energy to be constant, we must have that near coalescence points, the kinetic energy exactly cancels the Coulomb energy. A more formal treatment of this argument leads to the electron-electron Kato cusp condition,\cite{katoEigenfunctionsManyparticleSystems1957a}
\begin{equation}
    \label{eq:cusp}
    \left.\widetilde{\frac{\partial \Psi}{\partial r_{12}}}\right|_{r_{12}\to 0}
    = \frac 12 \Psi(r_{12}=0)
\end{equation}
where the tilde represents spherical averaging.

This cusp condition was also later generalised.\cite{packCuspConditionsMolecular1966,kurokawaChapterTwoGeneral2016}
Early literature on the subject suggested that the success of explicitly correlated methods were due to the superior description of short-range correlation effects, and in particular in their much more faithful capturing of the cusp conditions like equation \ref{sec:cusp}.\cite{roothaanAnalytical1960,watsonApproximate1960,weissConfiguration1961,schwartzGround1962}
However, further study found that the correlation error from a bad description of the wave function around a small sphere centred on the cusp is actually negligible.\cite{coulsonElectron1961,gilbertInterpretation1963,prendergastImpact2001,klopperR122007} Instead, the success of explicitly correlated methods is actually due to the superior description of the overall shape and size of the Coulomb hole, which has a radius on the order of the atomic radius.

To understand why gaussian-type basis sets fail so spectacularly at capturing the cusp behaviour, it is instructive to consider a simpler example, like that of approximating $|x|$ by its Fourier decomposition. Such an illustration is found in figure \ref{fig:cusp}. It is also worth noting that \glspl{STO} do not suffer as badly from this limitation,\cite{kongExplicitly2012} but as discussed in chapter \ref{sec:orbitals}, they are unsuccessful due to their lack of practicality.

% \todo{would be nice to think of other ways to add images}

\begin{figure}[htbp]
    \centering
    \includegraphics{figures/explicit/cusp.pdf}
    \caption{A toy example of the Coulomb cusp. Here, the Fourier expansion $x\approx\frac{\pi}{2} - \frac {4}{\pi} \sum_{n=1}^N\frac{\cos[(2n-1)x]}{(2n-1)^2}$ is plotted for a few values of $N$, including the exact solution. As can be seen, even for many terms, the Fourier expansion is a poor descripter in the cusp region. Indeed, the only way to describe it exactly is with an infinite number of terms.}
    \label{fig:cusp}
\end{figure}

\section{Hylleraas Methods}

Almost 30 years prior to Kato's landmark paper describing cusps in the analytical form for the wave function, there was already work being done to understand the significance of including $r_{12}$ terms in the wave function. Instead of being motivated by rigorous mathematics like Kato's, Slater was motivated by studies in the He atom. In particular, he tried to construct a wave function which faithfully represents both the core region as well as the Rydberg limit (i.e. a highly excited atom where the electron is very far from the nucleus).\cite{kongExplicitly2012,grynbergIntroduction2010,slaterCentral1928,slaterNormal1928} This led him to suggest multiplying the wave function by a factor
\begin{equation}
    \e^{-2(r_1+r_2)+r_{12}/2},
\end{equation}
which can easily be shown to satisfy the cusp equation \eqref{eq:cusp}.

However, the first successful explicitly correlated electronic structure calculation is typically attributed to Hylleraas,\cite{hattigExplicitly2012} where he aimed to improve convergence of orbital expansions for helium.\cite{hylleraasUeber1928,hylleraasNeue1929} In this method, the coordinates $s\mathdef r_1+r_2$, $t\mathdef r_1 - r_2$ and $u\mathdef r_{12}$ are used to construct the wave function
\begin{equation}
    \Psi_N(s,t,u) = \e^{-\alpha s}\sum_{k}^N c_ks^{l_k}t^{2m_k}u^{n_k}.
\end{equation}

In particular, using only three terms ($N=3$), and variationally optimising for the parameters $c_1,c_2,c_3$, Hylleraas was able to reach within 1.3 millihartree from the exact result.

Since then, there was rapid development on this approach and combining it with \gls{CI} (which came to be known as the CI-Hyl methods).
\cite{largo-cabrerizoHylleraasCI1987,jamesGround1933,kolosAccurate1964,perkinsAtomic1968,perkinsAtomic1969,simsCombined1971,simsOneCenter1971,claryHylleraastype1977,claryCIHylleraas1976} In CI-Hyl methods, the wave function is expanded as in \gls{CI},
\begin{equation}
    \Psi = \sum_k c_k \Phi_k
\end{equation}
where
\begin{equation}
    \Phi_k = \mathcal{A} r^{\nu_k}_{ij}\prod_i\chi_{k_i}(\bm x_i)
\end{equation}
where ${\chi_k}$ is a spin-orbital basis and $\mathcal{A}$ is the antisymmetriser operator.

However, CI-Hyl methods were to eventually fall out of favour. This is because the expansions involve exceedingly difficult integrals involving many electrons and over products of correlation factors. This significantly restricts the tractibility and scalability of the method, and it has since largely gone unused.

\section{Explicitly Correlated Gaussians}

Boys\cite{boysIntegral1960} and Singer\cite{singerUse1960} independently introduced gaussian basis functions with explicit correlation for calculations on molecules.\cite{mitroyTheory2013} These methods are referred to as \glspl{ECG}, or in the case of functions of two electrons, \glspl{GTG}. A spherical \gls{GTG} may be written (with nuclar coordinates $\bm R_1$ and $\bm R_2$) as
\begin{equation}
    g(\bm r_1, \bm r_2) = \exp(-\zeta_1|\bm r_1 - \bm R_1|^2 - \zeta_2|\bm r_2 - \bm R_2|^2 - \gamma|\bm r_1 - \bm r_2|^2).
\end{equation}
This can be interpreted as two $s$-type orbitals and an interelectronic correlation factor $\e^{-\gamma r_{12}}$.
While they do not have the correct cusp behaviour, much like how gaussian basis functions approximately capture the electron-nuclear cusp, a linear combination of \glspl{GTG} approximatelycapture the electron-electron cusp.

One major strength of this approach is that all integrals have closed-form algebraic expressions,\cite{lesterGaussian1964} and avoids nonlinear optimisation.\cite{bukowskiNew1994,perssonAccurate1996} \Gls{ECG} methods have been extended to post-Hartre-Fock methods, such as MP2\cite{panGaussian1970,panElectron1972}, and methods to avoid its many difficult integrals have also been developed.\cite{szalewiczNew1982,szalewiczAtomic1983,wenzelAtomic1986,szalewiczAtomic1984,tewWeak2007}

While not as popular as F12 methods (see section \ref{sec:f12}), \glspl{ECG} have been used for highly accurate variational calculations,\cite{korobovCoulomb2000} as well as for applications outside of standard electronic structure theory, such as bosons,\cite{vargaPrecise1995} positronium (a bound state of an electron and a positron),\cite{bubinGroundstate2011} and non-Born-Oppenheimer systems.\cite{stankeNonBornOppenheimer2009}

\section{F12 Methods}
\label{sec:f12}

The most influential class of explicitly correlated methods to date are the F12 methods. The core principle of these methods is to augment the wave function from a conventional (typically \gls{SD}) basis with an explicitly-correlated correction, called the F12 (or R12) correction. The original formulation\todo{cite Kutzelnigg} parametrised a two-electron system (such as helium) wave function as
\begin{equation}
    \label{eq:first-r12}
    \ket\Psi = (1+tQ_{12}F_{12}(r_{12}))\ket{\Psi_0}+\sum_{p}t_{p}\ket{\Psi_{p}}
\end{equation}
where $\Psi_0$ is a reference wave function (such as \gls{HF}), $\ket{\Psi_{p}}$ are excited-state \glspl{SD}, $t,t_{p}$ are parameters (``amplitudes'') to be optimised, and
\begin{equation}
    Q_{12} \mathdef \sum_{\alpha\beta}\ket{\alpha\beta}\bra{\alpha\beta}
\end{equation}
is referred to as a ``strong orthogonality projector'', which ensures the $F_{12}(r_{12})$ term is orthogonal to the reference and singly-excited determinants.
Here $\alpha,\beta$ refer to virtual orbitals in the formally complete basis. The fact that, due to $Q_{12}$, the $F_{12}(r_{12})$ term commutes with the standard excitation operators aids in including this explicit correlation into conventional electron correlation methods. This geminal term is added a correction to the standard wave function.

In the original formulation, $F_{12}(r_{12})=r_{12}$ (hence the name ``R12'' or ``F12'' for the more general methodology). However, another popular choice is a Slater-type geminal\todo{cite Tenno}, fitted to a linear combination of gaussians,\todo{citations}
\begin{equation}
    F_{12}(r_{12})=-\gamma^{-1}\e^{-\gamma r_{12}} = -\gamma^{-1}+r_{12}-\frac 12\gamma r_{12}^2+\cdots \approx \sum_i c_i\e^{-\alpha_ir_{12}^2}.
\end{equation}
The choice of the length scale $\gamma$ can either be optimised or (more typically) kept fix, although formally it is orbital dependent and a poor choice may result in a loss of accuracy.\supercite{tewRelaxing2018} Other choices for the correlation factor also exist.\todo{citations}

\subsection{MP2-F12}

Following the discussion in reference \citenum{shiozakiMultireference2013}, and adopting their notation, we consider the MP2-F12 method as an illustrative example.

An alternative derivation of the \gls{MP2} equations from section \ref{sec:perturbation-theory} is to minimise the Hylleraas functional\todo{citations}
\begin{equation}
    \label{eq:hyllerass-functional}
    E^{(2)} = \min_\Psi\bra{\Psi^{(1)}}(H_0-E^{(0)})\ket{\Psi^{(1)}}+2\bra{\Psi^{(1)}}H\ket{\Psi^{(0)}}
\end{equation}
where $H_0$ is the unperturbed Hamiltonians, and the $(i)$ superscript denotes the order of the correction, as introduced in section \ref{sec:perturbation-theory}.

The earliest generalisation of equation \ref{eq:first-r12} was an ansatz for MP2,\todo{citations}
\begin{equation}
    \label{eq:mp2-f12-firstorder}
    \ket{\Psi^{(1)}} = \frac 12\sum_{ij}\left( \sum_{ab}t_{ij}^{ab}\ket{\Psi_{ij}^{ab}} + \sum_{kl}t_{ij}^{kl}\sum_{\alpha\beta}\bra{\alpha\beta}Q_{12}F_{12}(r_{12})\ket{kl}\ket{\Psi_{ij}^{\alpha\beta}} \right),
\end{equation}
\begin{equation}
    \label{eq:mp2-f12-projector}
    Q_{12} = (1-o_1)(1-o_2) - v_1v_2
\end{equation}
where we define the one-electron projection operators
\begin{equation}
    o_m = \sum_i\ket{\phi_i(\bm r_m)}\bra{\phi_i(\bm r_m)}
\end{equation}
and
\begin{equation}
    v_m = \sum_a\ket{\phi_a(\bm r_m)}\bra{\phi_a(\bm r_m)}
\end{equation}
such that $o_m$ and $v_m$ project onto the occupied and virtual orbitals, respectively, with
\begin{equation}
    \bra{\phi_i(\bm r_m)}\Omega\ket{jk} = \int\d^3r_m\ \phi_i^*(\bm r_m)\Omega\phi_j(\bm r_1)\phi_k(\bm r_2)
\end{equation}
for operator $\Omega$. The first term in brackets in equation \ref{eq:mp2-f12-firstorder} are from the conventional MP2 wave function correction, and the extra term is a contraction of a formally-infinite set of double excitations, representing the F12 correction (again, orthogonal to the conventional wave function thanks to $Q_{12}$).

It was also shown that solving for the amplitudes $t_{ij}^{kl}$ can be avoided\todo{citation}, and they tend to also be more accurate\todo{citations}, derived from the cusp conditions,
\begin{equation}
    t_{ij}^{kl} = \frac 38\delta_{ik}\delta_{jl}+\frac 18\delta_{il}\delta_{jk}.
\end{equation}

\subsection{Many-Electron Integrals}

The Hylleraas functional, equation \ref{eq:hyllerass-functional}, will contain terms such as
\begin{equation}
    \sum_{\alpha\beta}\bra{\Phi_0}H\ket{\Psi_{ij}^{\alpha\beta}} \bra{\alpha\beta}Q_{12}F_{12}(r_{12})\ket{kl} = 2\bra{ij}r_{12}^{-1}Q_{12}F_{12}(r_{12})\ket{kl} - \bra{jl}r_{12}^{-1}Q_{12}F_{12}(r_{12})\ket{kl}.
\end{equation}
Plugging in the form for $Q_{12}$ from equation \ref{eq:mp2-f12-projector}, we get integrals of the form
\begin{equation}
    \bra{ij}r_{12}^{-1}o_1F_{12}(r_{12})\ket{kl}.
\end{equation}
For $F_{12}(r_{12})=r_{12}$, this is equal to
\begin{equation}
    \sum_o\bra{ijo}r_{12}^{-1}r_{23}\ket{olk},
\end{equation}
which is a three-electron integral. Clearly, with more operators multiplied together in the kernel of these integrals, we might expect even higher-order integrals. This may seem distrastrous, as this would be a massive bottleneck.

However, it was found that by insertion of the \gls{RI},

$\varphi \varPhi\Phi$

\todo{...}

\todo{also describe how the exponential Slater F12 function also improves the results}

\subsection{Higher-Order F12 Theories}

\todo{change in definition of excitation operators}

\subsection{Excited States}

\todo{different "target" -- don't remember, see paper}

% I think for this section I will largely be pulling from section 4 of \supercite{shiozakiMultireference2013}

% \subsection{A Historial Perspective: R12 Methods}
\todo{... see review papers haettig pg 30 for basic intro then 35 - 50 then excited state papers}

\todo{revisit}
brief mention of how it builds on gaussian-type geminals

% \subsection{}

\todo{make sure to include some excited state information}

\todo{mention but maybe don't go into too much detail on MR-F12 methods}

\section{The Transcorrelated Method}
\label{sec:tc}
\todo{brief recap of Boys-Handy method, some extensions by other people, and how it is used in our group}

Hirschfelder first introduced a similarity transform method \cite{hirschfelderRemoval1963}

then was worked on by Boys and Handy \todo{cite}, to be discussed here

uses ideas from \gls{VMC}, discussed more in section \ref{sec:vmc}.

recent renewal of interest (lots and lots of citations) with various applications, including DMRG, quantum computing, etc.

will talk in more detail about the Jastrow factor in section \ref{sec:jastrow} since it is also used in vmc

\todo{also mention biorthogonal orbitals etc.}

\todo{short proof of isospectrality}

\todo{reference section \ref{sec:variational_principle} and how it is not trivial to make TC variational since the proof of the variational principle uses the Hermiticity of the operator}

\subsection{The Method of Boys and Handy}

\subsection{Modern Resurgence}
\todo{Also cite Werner and quantum computing people}


\subsection{Comparison to F12/R12}

discuss additive vs multiplicative ansatz

have some performance comparison plots, maybe can ask permission to reproduce from Evelin or someone else

also discuss many-body integrals for f12 vs at most 3-body for tc


\todo{Maybe have a picture of basis set convergence? Maybe even just HF, compare F12 vs TC vs Hyl(?) vs standard}
