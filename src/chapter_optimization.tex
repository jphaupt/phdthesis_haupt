\chapter{Optimizing Jastrow Factors for the Transcorrelated Method}
  \label{chap:opt}
\todo{...}

This chapter is based in large part on the following paper:\\
\fullcite{hauptOptimizing2023}

Images have been reused from this paper (with permission).

\section{Introduction}

In this chapter, we investigate the use of flexible Jastrow factors and a novel optimisation strategy for use in \gls{TC} as introduced in section \ref{sec:tc}. As a brief recapitulation, the TC method amounts to a similarity transformation of the Hamiltonian $\hat H$, $\htc = \e^{-J}\hat H\e^J$. However, as this is a non-unitary trasformation, methods used to solve $\htc$ are in general not variational, and hence we are not guaranteed to converge to the \gls{CBS} limit from above. It is therefore important to choose $J$ wisely, as otherwise the method may be highly non-variational, and we may suffer from poor error cancellation.

\todo{...}

\section{Jastrow Factor}

In continuum quantum Monte Carlo methods, the Jastrow factor for a molecule is typically expressed as the sum of electron-electron, electron-nucleus, and electron-electron-nucleus terms,\footnote{Of course, these are not all the possible terms. We may, for example, also choose to include electron-nucleus-nucleus terms.}
\begin{equation}
    \label{eq:jastrow}
    J = \sum_{i<j}^Nu(r_{ij}) + \sum_i^N\sum_I^{N_A}\chi(r_{iI}) + \sum_{i<j}^N\sum_I^{N_A}f(r_{ij}, r_{iI}, r_{jI}),
\end{equation}
where $N_A$ is the number of nuclei, $N$ the number of electrons, and each of $u$, $\chi$, and $f$ are expressed as natural power expansions. That is,

\todo{...}

\section{Optimisation Strategy}

We optimise $J$ using \gls{VMC}.

\todo{...}

\subsection{Choosing an Appropriate Sample Size}

\todo{...}

\section{Results}
\todo{mention details about integration will be in a separate chapter}
\todo{mention we use a walker number extrapolation already described in another dissertation (cite Muhammedreza)}
\todo{...}

\subsection{Neglecting Three-Body Excitations}

\todo{...}
\todo{mention Pauli exclusion principle as an argument for why this is a valid approximation (maybe use a figure?)}

\subsection{Basis Set Convergence}

\todo{...}

\section{Conclusion and Outlook}
\todo{mention this is the way we now optimise Jastrow factors, but there is an important extension to the no-3-body approximation, xTC. Briefly describe.}
\todo{...}

\subsection{The xTC Approximation}

\todo{...}

\label{sec:xtc}
